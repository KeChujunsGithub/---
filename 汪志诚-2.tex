\section{均匀物质的热力学性质}
%%%%%%%%%%%%%%%%%%%%%%%%%%%%%%%%%%%%%%%%%%%%%%%%%%%
\subsection{内能、焓、自由能和吉布斯函数的全微分}










麦克斯韦关系
\begin{equation}
    \begin{aligned}
        \left( \frac{\partial T}{\partial V} \right) _S&=-\left( \frac{\partial p}{\partial S} \right) _V
\\
\left( \frac{\partial T}{\partial p} \right) _S&=\left( \frac{\partial V}{\partial S} \right) _p
\\
\left( \frac{\partial S}{\partial V} \right) _T&=\left( \frac{\partial p}{\partial T} \right) _V
\\
\left( \frac{\partial S}{\partial p} \right) _V&=-\left( \frac{\partial V}{\partial T} \right) _p
    \end{aligned}
\end{equation}




\subsubsection{推导:麦克斯韦关系}

1.
$U$作为$S$,$V$的函数$U=U(S,V)$,其全微分为
\begin{equation}
    \mathrm{d}U=\left( \frac{\partial U}{\partial S} \right) _V\mathrm{d}S+\left( \frac{\partial U}{\partial V} \right) _S\mathrm{d}V
\end{equation}
同时热力学基本方程
\begin{equation}
    \mathrm{d}U=T\mathrm{d}S-p\mathrm{d}V
\end{equation}
比较可得
\begin{equation}
    \left( \frac{\partial U}{\partial S} \right) _V=T,\left( \frac{\partial U}{\partial V} \right) _S=-p
\end{equation}
由于偏导数的次序可交换: 
\begin{equation}
    \frac{\partial ^2U}{\partial V\partial S}=\frac{\partial ^2U}{\partial S\partial V}
\end{equation}
即
\begin{equation}
    \left[ \frac{\partial}{\partial V}\left( \frac{\partial U}{\partial S} \right) _V \right] _S=\left[ \frac{\partial}{\partial S}\left( \frac{\partial U}{\partial V} \right) _S \right] _V
\end{equation}
且
\begin{equation}
    \left[ \frac{\partial}{\partial V}\left( \frac{\partial U}{\partial S} \right) _V \right] _S=\left( \frac{\partial T}{\partial V} \right) _S,\left[ \frac{\partial}{\partial S}\left( \frac{\partial U}{\partial V} \right) _S \right] _V=-\left( \frac{\partial p}{\partial S} \right) _V
\end{equation}
得到麦克斯韦关系
\begin{equation}
    \left( \frac{\partial T}{\partial V} \right) _S=-\left( \frac{\partial p}{\partial S} \right) _V
\end{equation}
2.
$H$作为$S$,$p$的函数$H=H(S,p)$,其全微分为
\begin{equation}
    \mathrm{d}H=\left( \frac{\partial H}{\partial S} \right) _{p}\mathrm{d}S+\left( \frac{\partial H}{\partial p} \right) _{S}\mathrm{d}p
\end{equation}
同时热力学基本方程
\begin{equation}
    \mathrm{d}H=T\mathrm{d}S+V\mathrm{d}p
\end{equation}
对比可得
\begin{equation}
    \left( \frac{\partial H}{\partial S} \right) _{p}=T,\left( \frac{\partial H}{\partial p} \right) _{S}=V
\end{equation}
由于偏导数的次序可交换: 
\begin{equation}
    \frac{\partial ^2H}{\partial p\partial S}=\frac{\partial ^2H}{\partial S\partial p}
\end{equation}
即
\begin{equation}
    \left[ \frac{\partial}{\partial p}\left( \frac{\partial H}{\partial S} \right) _p \right] _S=\left[ \frac{\partial}{\partial S}\left( \frac{\partial H}{\partial p} \right) _S \right] _p
\end{equation}
且
\begin{equation}
    \left[ \frac{\partial}{\partial p}\left( \frac{\partial H}{\partial S} \right) _p \right] _S=\left( \frac{\partial T}{\partial p} \right) _S,\left[ \frac{\partial}{\partial S}\left( \frac{\partial H}{\partial p} \right) _S \right] _p=\left( \frac{\partial V}{\partial S} \right) _p
\end{equation}
联立得到麦克斯韦关系
\begin{equation}
    \left( \frac{\partial T}{\partial p} \right) _S=\left( \frac{\partial V}{\partial S} \right) _p
\end{equation}
3.
$F$作为$T$,$V$的函数$F=F(T,V)$,其全微分为
\begin{equation}
    \mathrm{d}F=\left( \frac{\partial F}{\partial T} \right) _V\mathrm{d}T+\left( \frac{\partial F}{\partial V} \right) _T\mathrm{d}V
\end{equation}
同时热力学基本方程
\begin{equation}
    \mathrm{d}F=-S\mathrm{d}T-p\mathrm{d}V
\end{equation}
对比可得
\begin{equation}
    \left( \frac{\partial F}{\partial T} \right) _V=-S,\left( \frac{\partial F}{\partial V} \right) _T=-p
\end{equation}
由于偏导数的次序可交换
\begin{equation}
    \frac{\partial ^2F}{\partial V\partial T}=\frac{\partial ^2F}{\partial T\partial V}
\end{equation}
即
\begin{equation}
    \left[ \frac{\partial}{\partial V}\left( \frac{\partial F}{\partial T} \right) _V \right] _T=\left[ \frac{\partial}{\partial T}\left( \frac{\partial F}{\partial V} \right) _T \right] _V
\end{equation}
且
\begin{equation}
    \left[ \frac{\partial}{\partial V}\left( \frac{\partial F}{\partial T} \right) _V \right] _T=-\left( \frac{\partial S}{\partial V} \right) _T,\left[ \frac{\partial}{\partial T}\left( \frac{\partial F}{\partial V} \right) _T \right] _V=-\left( \frac{\partial p}{\partial T} \right) _V
\end{equation}
联立得到麦克斯韦关系
\begin{equation}
    \left( \frac{\partial S}{\partial V} \right) _T=\left( \frac{\partial p}{\partial T} \right) _V
\end{equation}


4.
$G$作为$T$, $p$的函数$G=G(T,p)$, 其全微分为
\begin{equation}
    \mathrm{d}G=\left( \frac{\partial G}{\partial T} \right) _p \mathrm{d}T+\left( \frac{\partial G}{\partial p} \right) _T \mathrm{d}p
\end{equation}
同时热力学基本方程
\begin{equation}
    \mathrm{d}G=-S\mathrm{d}T+V\mathrm{d}p
\end{equation}
对比可得
\begin{equation}
    \left( \frac{\partial G}{\partial T} \right) _p=-S,\left( \frac{\partial G}{\partial p} \right) _T=V
\end{equation}
由于偏导数的次序可交换
\begin{equation}
    \frac{\partial ^2G}{\partial T\partial p}=\frac{\partial ^2G}{\partial p\partial T}
\end{equation}
即
\begin{equation}
    \left[ \frac{\partial}{\partial p}\left( \frac{\partial G}{\partial T} \right) _p \right] _T=\left[ \frac{\partial}{\partial T}\left( \frac{\partial G}{\partial p} \right) _T \right] _p
\end{equation}
且
\begin{equation}
    \left[ \frac{\partial}{\partial p}\left( \frac{\partial G}{\partial T} \right) _p \right] _T=-\left( \frac{\partial S}{\partial p} \right) _T,\left[ \frac{\partial}{\partial T}\left( \frac{\partial G}{\partial p} \right) _T \right] _p=\left( \frac{\partial V}{\partial T} \right) _p
\end{equation}
联立得到麦克斯韦关系
\begin{equation}
    \left( \frac{\partial S}{\partial p} \right) _T=-\left( \frac{\partial V}{\partial T} \right) _p
\end{equation}


%%%%%%%%%%%%%%%%%%%%%%%%%%%%%%%%%%%%%%%%%%%%%%%%%%%
\subsection{}










%%%%%%%%%%%%%%%%%%%%%%%%%%%%%%%%%%%%%%%%%%%%%%%%%%%
\subsection{}





















%%%%%%%%%%%%%%%%%%%%%%%%%%%%%%%%%%%%%%%%%%%%%%%%%%%
\subsection{}




















%%%%%%%%%%%%%%%%%%%%%%%%%%%%%%%%%%%%%%%%%%%%%%%%%%%
\subsection{}















%%%%%%%%%%%%%%%%%%%%%%%%%%%%%%%%%%%%%%%%%%%%%%%%%%%
\subsection{}










%%%%%%%%%%%%%%%%%%%%%%%%%%%%%%%%%%%%%%%%%%%%%%%%%%%
\subsection{}












%%%%%%%%%%%%%%%%%%%%%%%%%%%%%%%%%%%%%%%%%%%%%%%%%%%
\subsection{}












%%%%%%%%%%%%%%%%%%%%%%%%%%%%%%%%%%%%%%%%%%%%%%%%%%%
\subsection{}
